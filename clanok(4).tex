\documentclass[10pt]{article}
\usepackage{titlesec}
\usepackage{biblatex}

\addbibresource{literatura.bib}

\title{ Innovations in Information Retrieval: Voice and Visual Search\thanks{Semester Project in the Methods of Engineering Work, Academic Year 2023/2024, Supervisor: - }}
\author{Tikhoblazhenko Alena\\[2pt]
	{\small Slovak Technical University in Bratislava}\\
	{\small Faculty of Informatics and Information Technologies}\\
	{\small \texttt{alanabluecake@gmail.com}}
	}
\date{09. October 2023} 

\begin{document}
\maketitle

\section*{Abstract}
With the development of technology, searching for information has become an integral part of everyday life. And new methods of searching for information began to appear, such as voice and visual search. This article examines the current trends and prospects of these technologies. The objective of this article is to examine the latest innovations in voice and visual information retrieval, their impact on user experience, and their implications for future technological development. The key objectives of this article are: analyze key principles and technologies, study technical aspects, study the advantages and disadvantages, summarize the impact of technology on people and predict the further development of these technologies. Innovations in voice and visual search are making it easier and more accessible for a diverse group of people to find information. They also become indispensable components of our lives.

\section{Introduction}
----

\section{Voice and Visual Search Technologies}
----

\section{Technical Aspects}
----

\section{Advantages and Disadvantages}
----

\section{Impact on User Experience}
----

\section{Future Developments}
----

\printbibliography[title={Literature}]
\end{document}